%!TEX root = main.tex

% -----------------------------------------------------------------------------
% Document layout standard
%
\documentclass[11pt, a4paper, oneside]{article}
%\documentclass[11pt, a4paper, twoside]{article}


% -----------------------------------------------------------------------------
% LaTeX text output style
%
\usepackage[utf8]{inputenc}					% *input* encoding support
\usepackage{fontenc}               			% *font* encoding support
\renewcommand{\familydefault}{\rmdefault}   % font type - roman
%\renewcommand{\familydefault}{\sfdefault}   % font type - sans serif
%\renewcommand{\familydefault}{\ttdefault}   % font type - monospace
\usepackage[UKenglish]{babel}				% local language support

%
% LaTeX page layout style and geometry
%
%\usepackage{geometry}               % manage page layouts
%\usepackage{fancyhdr}               % caption style, headers and page styles
%\usepackage{footnote}               % handle footnotes
%\usepackage{setspace}               % set paragraph spacing
%\usepackage{pdflscape}              % PDF support to the environment landscape

%
% LaTeX support for the ifdf conditional
%
\usepackage{ifpdf}

%
% LaTeX document cross rerefence commands
%
\usepackage[numbers,sort&compress]{natbib}  % bibliography manager
\usepackage[
	pdftex,
	colorlinks,
	bookmarksopen,
	bookmarksnumbered,
	citecolor=red,
	urlcolor=red]{hyperref}        	% hyper reference

%\ifpdf
%\hypersetup{
%pdftitle={Untitled},
%pdfauthor={Unknown},
%colorlinks=true,    				% false: boxed links; true: colored links
%linkcolor=red,						% color of internal links
%citecolor=red,						% color of links to bibliography
%filecolor=magenta,          		% color of file links
%urlcolor=cyan               		% color of external links
%}
%\fi

%\usepackage{authblk}				% author blocks


% -----------------------------------------------------------------------------
% Maths packages
%
\usepackage{amsmath}                % amsmath package
\usepackage{amsfonts}               % amsmath fonts
\usepackage{amssymb}                % amsmath symbole
\usepackage{mathrsfs}               % other mathematical symbols
\usepackage{enumerate}              % manage enumerate lists
\everymath{\displaystyle}

%\usepackage{dsfont}                 % doublestroke math symbols
%\usepackage{esint}                  % variable size math operators
%\usepackage{euscript}               % Euler script symbols
%\usepackage{textcmds}               % commands for text symbols

%\usepackage[version=3]{mhchem}    	% typeset chemical formulae


% -----------------------------------------------------------------------------
% Graphics (Figures and Tables) packages
%
\usepackage{graphicx}               % manage pictures
\usepackage{subfig}                 % subfigures and subtables within floats
\usepackage{caption}                % figure captions
\usepackage{floatrow}               % place a caption beside a float
\usepackage{multirow}               % spanning rows in tables
\usepackage{float}                  % figures with borders
\floatstyle{plaintop}               % captions above the float (or plain, boxed, ruled)
\restylefloat{table}                % apply the style to tables
\usepackage[usenames]{color}        % used for font color
\usepackage[svgnames]{xcolor}       % colors by their 'svgnames'


% -----------------------------------------------------------------------------
% User defined commands
%
% 	\newcommand defines a new command - it is an error if it is already defined.
% 	\renewcommand redefines a predefined command - it is an error if it is not
% 	yet defined.
% 	\providecommand defines a new command if it isn't already defined.
%
% User defined equations, figures and tables.
%
% Long equation environment
%
%	\begin{widetext}
%	$$\mbox{put long equation here}$$
%	\end{widetext}
%
% Figures should be put into the text as floats.
%
%	\begin{figure}
%	\includegraphics{}%
%	\caption{\label{}}%
%	\end{figure}
%
% Tables may be be put in the text as floats. Insert the column specifiers
% (l, r, c, d, etc.) in the empty braces of the \begin{tabular}{} command.
%
%	\begin{table}
%	\caption{\label{} }
%	\begin{tabular}{}
%	\end{tabular}
%	\end{table}
%
% Reference using the \cite, \citep, \ref, and \label commands
%
\newcommand{\ii}{\mathrm{i}\,}                          % exponential symbol
\newcommand{\ee}{\mathrm{e}\,}                          % exponential symbol
\newcommand{\erf}{\mathrm{erf}\,}                       % error function
\newcommand{\erfc}{\mathrm{erfc}\,}                     % comp error function
\newcommand{\abs}[1]{\left|#1\right|}                   % absolute value
\newcommand{\dd}[1]{\,\mathrm{d}#1\,}                   % integral d symbol

\newcommand{\bb}[1]{\left(#1\right)}                    % normal brackets
\newcommand{\bba}[1]{\left<#1\right>}                   % angle brackets
\newcommand{\bbm}[1]{\left|#1\right|}                   % modulus brackets
\newcommand{\bbs}[1]{\left[#1\right]}                   % square brackets
\newcommand{\bbc}[1]{\left\{#1\right\}}                 % curly brackets

\newcommand{\fracd}[2]{\frac{\mathrm{d}\,#1}{\mathrm{d}\,#2}}           % derivative fraction
\newcommand{\fracp}[2]{\frac{\partial\,#1}{\partial\,#2}}               % partial fraction
\newcommand{\fracv}[2]{\frac{\delta\,#1}{\delta\,#2}}                   % variational fraction
\newcommand{\fracs}[2]{\left.#1\middle/#2\right.}                       % horizontal fraction
\newcommand{\fracps}[2]{\left.\partial\,#1\middle/\partial\,#2\right.}  % horizontal partial fraction

\newcommand{\lint}[2]{\int\limits_{#1}^{#2}}            % single int limits
\newcommand{\liint}[2]{\iint\limits_{#1}^{#2}}          % double int limits
\newcommand{\liiint}[2]{\iiint\limits_{#1}^{#2}}        % triple int limits
\newcommand{\lsum}[2]{\sum\limits_{#1}^{#2}\,}          % single int limits

\newcommand{\etal}{\textit{et al}.}                     % et al citation
\newcommand{\fig}[1]{Fig.~\ref{#1}}                     % figure reference
\newcommand{\Fig}[1]{Figure~\ref{#1}}                   % figure reference
\newcommand{\eqn}[1]{Eq.~(\ref{#1})}                    % equation reference
\newcommand{\Eqn}[1]{Equation~(\ref{#1})}               % equation reference
\newcommand{\tab}[1]{Table~(\ref{#1})}                  % table reference
\newcommand{\secn}[1]{Section~(\ref{#1})}               % section reference

%
% User defined colors
%
\newcommand{\red}[1]{\textcolor{red}{#1}}
\newcommand{\green}[1]{\textcolor{green}{#1}}
\newcommand{\blue}[1]{\textcolor{blue}{#1}}
\newcommand{\cyan}[1]{\textcolor{cyan}{#1}}
\newcommand{\yellow}[1]{\textcolor{yellow}{#1}}
\newcommand{\magenta}[1]{\textcolor{magenta}{#1}}
\definecolor{dblue_color}{RGB}{0,63,119}
\newcommand{\dblue}[1]{\textcolor{dblue_color}{#1}}

%
% Keywords command
%
\providecommand{\keywords}[1]
{
  \small
  \textbf{\textit{Keywords --}} #1
}

%
% Email command
%
\providecommand{\email}[1]
{
  \small
  \footnote{#1}
}




% -----------------------------------------------------------------------------
%
% Begin document
%
\begin{document}

%
% Document title and maketitle must precede title, authors and abstract
%
\title{Experiments on nonlinear dynamics and chaos}
\date{\today}
\maketitle     			% after title, authors, abstract and \pacs
%\thispagestyle{empty}	% <empty, plain> do not show page numbers

%
% Document authors
%
\begin{center}
\author{Carlos Correia}
\email{ubikoo@mailfence.com}
\end{center}

%
% Document abstract
%
%\begin{abstract}
%Abstract.
%\end{abstract}
%\keywords{Key 1, Key 2}	% use showkeys class option if keyword
%



% -----------------------------------------------------------------------------
% Document Body
%
\section{Introduction}
\label{sec:intro}
These notes are a research record of the computational experiments on nonlinear dynamics and chaos.
They are mostly based on the books \textit{Nonlinear Dynamics and Chaos} by J.M.T. Thompson, H.B. Stewart~\citep{thompson2002nonlinear}.
The other two foundational references are
\textit{Time Reversibility, Computer Simulation, Algorithms, Chaos} by W.G. Hoover, and C.G. Hoover~\citep{hoover2012time} and \textit{Practical Numerical Algorithms for Chaotic Systems} by T.S. Parker, and L. Chua~\citep{parker2012practical}.;

The notes are organised as a sparse set of self contained computational experiments.


% -----------------------------------------------------------------------------
% Acknowledgements
\section{Non linear phenomena}
\label{sec:1}
%
%\subsection{Undamped, unforced linear oscillator}
%
%\subsection{Undamped, unforced nonlinear oscillator}
%
%\subsection{Damped, unforced linear oscillator}
%
%\subsection{Damped, unforced nonlinear oscillator}
%
%\subsection{Forced linear oscillator}
%
%\subsection{Forced nonlinear linear oscillator: periodic attractors}
%
%\subsection{Forced nonlinear linear oscillator: chaotic attractor}
%




% -----------------------------------------------------------------------------
% Acknowledgements
%
%\section*{Acknowledgments}
%
% External bibliography
%
\bibliographystyle{plain}	% <plain, phaip>
\bibliography{references}

% Embedded bibliography
%
%\bibliographystyle{plain}	% plain, phaip

%%merlin.mbs apsrev4-1.bst 2010-07-25 4.21a (PWD, AO, DPC) hacked
%%Control: key (0)
%%Control: author (8) initials jnrlst
%%Control: editor formatted (1) identically to author
%%Control: production of article title (-1) disabled
%%Control: page (0) single
%%Control: year (1) truncated
%%Control: production of eprint (0) enabled
%\begin{thebibliography}{1}%
%\makeatletter
%\providecommand \@ifxundefined [1]{%
% \@ifx{#1\undefined}
%}%
%\providecommand \@ifnum [1]{%
% \ifnum #1\expandafter \@firstoftwo
% \else \expandafter \@secondoftwo
% \fi
%}%
%\providecommand \@ifx [1]{%
% \ifx #1\expandafter \@firstoftwo
% \else \expandafter \@secondoftwo
% \fi
%}%
%\providecommand \natexlab [1]{#1}%
%\providecommand \enquote  [1]{``#1''}%
%\providecommand \bibnamefont  [1]{#1}%
%\providecommand \bibfnamefont [1]{#1}%
%\providecommand \citenamefont [1]{#1}%
%\providecommand \href@noop [0]{\@secondoftwo}%
%\providecommand \href [0]{\begingroup \@sanitize@url \@href}%
%\providecommand \@href[1]{\@@startlink{#1}\@@href}%
%\providecommand \@@href[1]{\endgroup#1\@@endlink}%
%\providecommand \@sanitize@url [0]{\catcode `\\12\catcode `\$12\catcode
%  `\&12\catcode `\#12\catcode `\^12\catcode `\_12\catcode `\%12\relax}%
%\providecommand \@@startlink[1]{}%
%\providecommand \@@endlink[0]{}%
%\providecommand \url  [0]{\begingroup\@sanitize@url \@url }%
%\providecommand \@url [1]{\endgroup\@href {#1}{\urlprefix }}%
%\providecommand \urlprefix  [0]{URL }%
%\providecommand \Eprint [0]{\href }%
%\providecommand \doibase [0]{http://dx.doi.org/}%
%\providecommand \selectlanguage [0]{\@gobble}%
%\providecommand \bibinfo  [0]{\@secondoftwo}%
%\providecommand \bibfield  [0]{\@secondoftwo}%
%\providecommand \translation [1]{[#1]}%
%\providecommand \BibitemOpen [0]{}%
%\providecommand \bibitemStop [0]{}%
%\providecommand \bibitemNoStop [0]{.\EOS\space}%
%\providecommand \EOS [0]{\spacefactor3000\relax}%
%\providecommand \BibitemShut  [1]{\csname bibitem#1\endcsname}%
%\let\auto@bib@innerbib\@empty
%%</preamble>
%\bibitem [{\citenamefont {Adams}(1995)}]{adams1995hitchhiker}%
%  \BibitemOpen
%  \bibfield  {author} {\bibinfo {author} {\bibfnamefont {D.}~\bibnamefont
%  {Adams}},\ }\href {http://books.google.com/books?id=W-xMPgAACAAJ} {\emph
%  {\bibinfo {title} {The Hitchhiker's Guide to the Galaxy}}}\ (\bibinfo
%  {publisher} {San Val},\ \bibinfo {year} {1995})\BibitemShut {NoStop}%
%\end{thebibliography}%

\end{document}
